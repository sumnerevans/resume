% !TEX program = xelatex
\documentclass[10pt,letterpaper]{article}
\usepackage[margin=0.55in]{geometry}
\usepackage{graphicx}
\usepackage[pdfnewwindow=true]{hyperref}
\usepackage{sectsty}
\usepackage{enumitem}
\usepackage{titlesec}
\usepackage[usenames,dvipsnames,svgnames,table]{xcolor}
\usepackage{forkawesome}
\usepackage{fancyhdr}
\usepackage{sectsty}
\usepackage{fontspec}
\usepackage[document]{ragged2e}
\usepackage{multicol}

\tolerance=1
\emergencystretch=\maxdimen
\hyphenpenalty=10000
\hbadness=10000

\pagestyle{fancy}

\definecolor{header-blue}{HTML}{1375CF}
\renewcommand{\headrulewidth}{0pt}

\setlength{\parindent}{0pt}
\setlength{\parskip}{0pt}
%\sectionfont{\color{header-blue}\fontsize{12}{0}\selectfont\MakeUppercase}
\sectionfont{\fontsize{12}{0}\selectfont\MakeUppercase}
\titlespacing*{\section}{0pt}{10pt}{6pt}
\pagenumbering{gobble} % No page number

\setlist[itemize]{itemsep=-1pt,leftmargin=13pt}

\setmainfont{Iosevka}

\usepackage{fancyhdr}
\pagestyle{fancy}
\cfoot{\textit{Last Updated: \today}}

\begin{document}

\begin{minipage}[b][][b]{0.5\linewidth}
    {\huge\textbf{Jonathan Sumner Evans}}

    \vspace{5pt}
    \href{mailto:resume@sumnerevans.com}{\faEnvelope\ resume@sumnerevans.com}
    $\bullet$\ \href{https://matrix.to/#/@sumner:nevarro.space}{\faMatrixOrg\ @sumner:nevarro.space}
\end{minipage}\hfill
\begin{minipage}[b][][b]{0.278\linewidth}
    \href{https://sumnerevans.com}{\faGlobe\ sumnerevans.com} \\
    \href{https://www.linkedin.com/in/sumnerevans}{\faLinkedin\ linkedin.com/in/sumnerevans} \\
    \href{https://github.com/sumnerevans}{\faGithub\ github.com/sumnerevans}
\end{minipage}
\rule{\textwidth}{0.5pt}

\section*{\faBriefcase\ Work Experience}
{\fontsize{11}{0}
\textbf{Software Engineer, Beeper} --- Automattic --- Remote}
\hfill \textit{April 2024 - Present}
\begin{itemize}
    \item Saved {\textasciitilde}2TB of RAM by rewriting the
        \href{https://github.com/mautrix/telegramgo}{Telegram}
        to \textbf{Matrix} bridge from Python to \textbf{Go}.
    \item Implemented the cryptographic key infrastructure necessary for message
        key backups and interactive device verification in
        \href{https://github.com/mautrix/go}{mautrix-go} by utilizing the
        standard Go \textbf{cryptography libraries}.
    \item Implemented media upload/download and interactive device verification
        in the Beeper client SDK written in \textbf{Go} which is being used in
        the next generation Beeper clients.
\end{itemize}

{\fontsize{11}{0}
\textbf{Software Engineer} --- Beeper (acquired by Automattic) --- Remote}
\hfill \textit{July 2021 - April 2024}
\begin{itemize}
    \item Scaled our backend infrastructure from handling \textless 1,000 users
        to \textgreater 100,000 users by \textbf{sharding} traffic from
        high-volume bridges to a separate \textbf{Go} service called
        \href{https://github.com/sumnerevans/hungryserv-presentation}{\textit{Hungryserv}}
        in a backwards-compatible, transparent manner. I created the initial
        proof of concept and then continued as a core member of the 3-member
        team that productionized the project over a four-month period.
    \item \textbf{Reverse-engineered} and implemented features for
        \href{https://blog.beeper.com/p/introducing-beeper-mini-get-blue}{\textit{Beeper Mini}}
        (iMessage on Android) including media, tapbacks, typing indicators, read
        receipts, edits, unsends, link previews, and chat metadata changes.
    %\item Streamlined our support team's ability to chat directly with users by
    %    building a
    %    \href{https://github.com/beeper/chatwoot}{Chatwoot bot} in
    %    \textbf{Go}.
    \item Measured message send \textbf{latency and reliability} by
        instrumenting bridge \textbf{metrics}. Built a Dockerized \textbf{Go}
        service to process those metrics and send them to BigQuery.
    \item \textbf{Reverse-engineered} the LinkedIn Messaging API and implemented a
        \href{https://github.com/beeper/linkedin}{LinkedIn} to Matrix bridge in
        \textbf{Python}.
    \item Designed a framework for importing users' chat history, and
        implemented it in the
        \href{https://github.com/mautrix/whatsapp}{WhatsApp},
        \href{https://github.com/mautrix/facebook}{Facebook},
        \href{https://github.com/mautrix/instagram}{Instagram}, and
        \href{https://github.com/mautrix/telegramgo}{Telegram} bridges.
    % \item Building and modifying various services for managing our
    %     \textbf{Kubernetes}-based infrastructure.
\end{itemize}

{\fontsize{11}{0}
\textbf{Adjunct Professor} --- Colorado School of Mines --- Golden, CO}
\hfill \textit{Aug. 2018 - Dec. 2024}
\begin{itemize}
    \item \textbf{Algorithms} (4$\times$) ---
        advanced data structures, graph algorithms, dynamic programming,
        NP-completeness
    \item \textbf{Principles of Programming Languages} (4$\times$) ---
        functional programming, parsers, language implementation
    \item \textbf{Computer Organization} (1$\times$) ---
        RISC-V assembly, processor design, memory hierarchy
    \item \textbf{Advanced Computer Architecture} (1$\times$) ---
        cache coherence, virtual memory, branch prediction
\end{itemize}

%\hspace{0.01\linewidth}%
%\begin{minipage}[t]{0.45\linewidth}
%    \textit{\underline{Responsibilities:}}
%    \begin{itemize}[topsep=5pt]
%        \item \textbf{Lecture} to classes of 60+ students.
%        \item Hold \textbf{office hours} to assist students on projects and
%            homework.
%        \item Coordinate course content and grading policies with TAs and other
%            instructors.
%        \item \textbf{Design homework} assignments and worksheets.
%        \item \textbf{Develop new projects} with starter code.
%    \end{itemize}
%\end{minipage}\hfill%
%\begin{minipage}[t]{0.54\linewidth}
%    \textit{\underline{Courses:}}
%    \begin{itemize}[topsep=5pt]
%        \item \textbf{Algorithms} (3$\times$) \\
%            \textit{graph algorithms, dynamic programming, NP-completeness}
%        \item \textbf{Computer Organization} (1$\times$) \\
%            \textit{RISC-V assembly, processor design, memory hierarchy}
%        \item \textbf{Principles of Programming Languages} (4$\times$) \\
%            \textit{functional programming, language implementation}
%        \item \textbf{Advanced Computer Architecture} (1$\times$) \\
%            \textit{cache coherence, virtual memory, branch prediction}
%    \end{itemize}
%\end{minipage}

\vspace{3pt}
{\fontsize{11}{0}
\textbf{Software Engineer} --- The Trade Desk --- Denver, CO}
\hfill \textit{June 2019 - July 2021}
% \begin{itemize}
%     \item I was responsible for building features related to Connected TV across
%         the entire stack including the \textbf{C\#} backend, the \textbf{React}
%         frontend, and the \textbf{SQL Server} database.
% \end{itemize}

% \textbf{Software Engineering Intern} --- Pivotal --- Denver, CO
% \hfill May 2018 - Aug. 2018

% \textbf{Teachers Assistant (Data Structures)} --- Colorado School of Mines ---
% Golden, CO \hfill Aug. 2017 - May 2018

% \textbf{Software Development Intern} --- Kenzan --- Denver, CO
% \hfill June 2017 - Aug. 2017

\vspace{3pt}
{\fontsize{11}{0}
\textbf{Software Developer} --- Can/Am Technologies, Inc. --- Lakewood, CO}
\hfill \textit{Feb. 2013 - Aug. 2016}
%\begin{itemize}
%    \item Built a Check 21 ICL (Image Cash Letter, used for electronic check
%        deposits) file generator in \textbf{C\#}.
%\end{itemize}

% \begin{itemize}
%     \item Designed and built new features for Teller, an enterprise
%         point-of-sale system for municipal governments.
%     \item Implemented plugins to integrate Teller with external vendors
%         including Bank of America and Tyler Tech.
%     \item Worked in an Agile environment on \textbf{C\#} and \textbf{JavaScript}
%         codebases.
%     \item Helped transition Teller from a Windows Desktop application to a
%         web-based application.
% \end{itemize}

\section*{\faBook\ Education}
{\fontsize{11}{0}
\textbf{Colorado School of Mines} --- Golden, CO --- B.S. + M.S. Computer Science}
\hfill \textit{July 2016 - May 2019}
\begin{itemize}
    \item Outstanding Graduating Senior for Computer Science
    \item Chair of Mines ACM, Service Chair of Tau Beta Pi, \textbf{Linux} Help
        Guru of Mines Linux Users Group (LUG)
\end{itemize}

\section*{\faUsers\ Talks and Presentations}

{\fontsize{11}{0}\href{https://sumnerevans.com/posts/matrix/cryptographic-key-infrastructure}{%
\textbf{Matrix Cryptographic Key Infrastructure}}
--- Matrix Conference}
\hfill \textit{Sept. 2024}
\begin{itemize}
    \item Provided an overview of how key sharing, key backup, device and user
        verification, and secret storage operate within Matrix to provide
        cryptographically secure messaging features.
\end{itemize}

{\fontsize{11}{0}\href{https://github.com/sumnerevans/hungryserv-presentation}{%
\textbf{Hungryserv: A Homeserver Optimized for Unfederated Use-Cases}}
--- Berlin Matrix Summit}
\hfill \textit{Aug. 2022}
\begin{itemize}
    \item Discussed a Matrix-compatible homeserver that Beeper uses to handle
        unfederated bridge traffic.
\end{itemize}

\section*{\faCode\ Notable Projects}
{\fontsize{11}{0}
\textbf{Nix Home Manager} (Maintainer) ---
\href{https://github.com/nix-community/home-manager}{github.com/nix-community/home-manager} --- MIT}
\hfill \textit{April 2021 - Present}
% \begin{itemize}
%     \item I am a maintainer with write-access to the project repository. I
%         review and merge contributions to the project as well as contribute
%         fixes myself.

%         % Nix Home Manager provides a system for managing a user environment
%         % using the \textbf{Nix} package manager.
%     % \item I am a maintainer with write access to the project. I review and merge
%         % contributions to the project.
% \end{itemize}

\vspace{3pt}
{\fontsize{11}{0}
\textbf{Sublime Music} (Author) ---
\href{https://github.com/sublime-music/sublime-music}{github.com/sublime-music/sublime-music} --- GPLv3}
\hfill \textit{Nov. 2018 - Dec. 2024}
%\vspace{5pt}
\begin{itemize}
    \item A native Subsonic music server client for Linux built using
        \textbf{GTK} and \textbf{Python}.
     %\item Allows users to connect to multiple Subsonic API-compliant servers and
     %    browse and play songs from them.
     %\item \textit{Features Include:} playback through Chromecast devices, DBus
     %    MPRIS integration, play queue, offline mode.
\end{itemize}

%\section*{\faTrophy\ Prizes and Awards}
%\begin{itemize}
%    % \item \textit{Second Place} at HackCU V with a team of Freshmen + myself
%    %     for
%    %     \href{https://sumnerevans.com/pages/portfolio.html#MLocate}{\textit{MLocate}}
%    %     (February 2019)
%    \item \textit{First Place} at the 2018 Facebook Global Hackathon Finals at
%        Facebook HQ for
%        \href{https://sumnerevans.com/pages/portfolio.html#HypAR-Map}{\textit{HypAR Map}}
%        \hfill \textit{November 2018}
%%     % \item \textit{Best use of GCP, Facebook Best Social Good Hack} at MHacks for
%%     %     \href{https://sumnerevans.com/pages/portfolio.html#Datanium}{\textit{Datanium}}
%%     %     (October 2018)
%    \item \textit{Fourth Place} in 2018 Regional ACM International Collegiate
%        Programming Contest (ICPC) \hfill \textit{November 2018}
%%     % \item \textit{First Place} at Google Games in Boulder (April 2018)
%%     % \item \textit{Judges Favorite, Best Use of AWS, Dish Network sponsor prize}
%%     %     at HackCU IV for
%%     %     \href{https://sumnerevans.com/pages/portfolio.html#Wii-Track}{\textit{Wii Track}}
%%     %     (February 2018)
%%     % \item \textit{Grand Prize} at the Xilinx Pynq Hackathon for
%%     %     \href{https://sumnerevans.com/pages/portfolio.html#Parqyng-Lots}{\textit{Parqyng Lots}}
%%     %     (October 2017)
%\end{itemize}

% \textbf{Visplay} ---
% \href{https://gitlab.com/ColoradoSchoolOfMines/visplay}{gitlab.com/ColoradoSchoolOfMines/visplay}
% --- GPLv3 \hfill \textit{February 2018 - May 2020}
% \begin{itemize}
%     \item Mines ACM project to create a digital signage system with a dynamic,
%         hierarchical configuration system.
%     \item Worked on the design of the overall \textbf{system's architecture}.
%     \item Contributed in a \textbf{project management} role, and acted as
%         \textbf{technical lead} for configuration GUI.
%     \item I was a core developer of the \textbf{Python} backend and the
%         \textbf{CI/CD} infrastructure for the project.
% \end{itemize}

% \textbf{HypAR Map} ---
% \href{https://gitlab.com/ColoradoSchoolOfMines/facebook-hackathon}{gitlab.com/ColoradoSchoolOfMines/facebook-hackathon}
% --- AGPLv3 \hfill \textit{November 2018}
% \begin{itemize}
%     \item Indoor navigation application which uses \textbf{AR} and
%         \textbf{Structure from Motion} to pinpoint the user's location on a
%         picture of a building map.
%     \item Worked on the image import functionality and connecting all of the
%         components together.
%     \item \textit{Awards:} \textbf{First Place} at the 2018 Facebook Global
%         Hackathon Finals at Facebook HQ.
% \end{itemize}

% \textbf{Wii-Track} ---
% \href{https://github.com/ColoradoSchoolOfMines/wii-track}{github.com/ColoradoSchoolOfMines/wii-track}
% --- GPLv3 \hfill \textit{February 2018}
% \begin{itemize}
%     \item Distributed inventory tracking using IoT technologies. Backed by
%         \textbf{AWS Lambda} and \textbf{DynamoDB}.
%     \item Worked on designing the overall system's \textbf{architecture}, set up
%         the database, and implemented a Lambda function to identify packages by
%         weight.
%     \item \textit{Awards:} \textbf{Judges Favorite}, Best Use of AWS, and Dish
%         Network Challenge winner at HackCU IV.
% \end{itemize}

% \textbf{Virtual Reality Final Project} ---
% \href{https://github.com/CSM-Dream-Team/final-project}{github.com/CSM-Dream-Team/final-project}
% --- GPLv3 \hfill \textit{Aug. 2017 - Dec. 2017}
% \begin{itemize}
%     \item Final project from an independent study in \textbf{virtual reality}
%         under the supervision of Dr. Paone.
%     \item We developed a new UI architecture for virtual reality called
%         \textit{Deferred Immediate Mode}.
% \end{itemize}

\end{document}
